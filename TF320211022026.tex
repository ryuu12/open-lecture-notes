\documentclass{article}
\usepackage{graphicx} % Required for inserting images

\title{Lecture Note Week 1}
\author{TF3202 Fisika Bangunan}

\begin{document}

\maketitle

\section{Pengantar}
Teknik Fisika mendekati ilmu fisika bangunan dari sisi non-struktur, seperti analisis termal, pencahayaan, akustik, dan aliran udara. Artinya, pendekatannya akan melalui tentang kesehatan dan kenyamanan. Selain itu, hal lain yang menjadi konsentrasi fisika bangunan adalah sistem pendukung bangunan, seperti material dan utilitas/komponen bangunan.
\subsection{Silabus}
Mata kuliah ini mencakup topik yang terkait dengan evaluasi dan pengukuran aspek-aspek fisis bangunan. Seperti,
\begin{itemize}
    \item Kinerja dan kenyamanan termal: iklim, respon termal bangunan, kenyamanan termal, serta pengendalian termal pasif dan aktif pada bangunan.
    \item Kinerja dan kenyamanan pencahayaan: fotometri dan perambatan cahaya, penglihatan dan kenyamanan visual, perhitungan iluminansi, serta pencahayaan alami siang hari dalam bangunan.
    \item Kinerja dan kenyamanan akustik: perambatan suara, pendengaran dan kenyamanan audial, pengendalian bising pada bangunan, serta akustik ruang.
\end{itemize}
Artinya, fisika bangunan adalah bagaimana kita dapat mencapai \textit{wellness} penghuni dan disaat yang bersamaan memperhatikan konsumsi energi bangunan dengan cara mengatur besaran-besaran fisis yang ada pada bangunan tersebut.

\section{Iklim}
Kenapa iklim menjadi suatu \textit{key driver} seorang \textit{engineer} atau arsitek ketika mendesain suatu bangunan? Perumpamaan iklim dengan kenyamanan adalah seperti air di dalam kolam. Setiap ikan memiliki habitat masing-masing atau kolam masing-masing. Iklim pun seperti habitat bagi manusia dan bangunan. Tentu, iklim tidak dapat direkayasa, karena itu yang paling realistis adalah merekayasa bangunan.\\\\
Menariknya, iklim ini berpola. Walaupun tidak dapat diprediksi secara persis, tapi ada pola yang terbentuk pada iklim dalam waktu satu tahun. Tentunya, dalam perubahan iklim, nantinya akan berpengaruh terhadap keputusan-keputusan seperti pemilihan bahan, pemasangan shading, dan sebagainya.\\\\
Apabila sistem bangunan dibayangkan dengan model sistem yang memiliki input dan output, maka iklim ini adalah bagian dari input terhadap sistem bangunan. Dalam proses simulasi bangunan, iklim biasanya termasuk dalam \textit{weather data}. Isinya berupa temperatur udara, RH, dan kecepatan angin.\\\\
Kita ambil bangunan di negara-negara tropis. Terlihat bahwa bukaan dari bangunan sangat besar dan banyak. Berbanding terbalik dengan negara skandinavian, mereka bukaannya kecil sekali. Kenapa? Karena di negara tropis, \textit{air exchange rate} lebih efektif sehingga lebih sering memanfaatkan udara alami.\\\\
Lalu bagaimana dengan output dari sistem bangunan? Outputnya berupa konsumsi energi dan \textit{wellness} penghuni bangunan. Dalam efisiensi energi, ketika diminta untuk memilih di bagian \textit{production} atau \textit{end-use}, maka yang paling signifikan adalah sektor \textit{end-use}. Sederhananya, antara tempat konversi energi dan rumah atau bangunan penerima/pengguna energi, dampaknya akan lebih signifikan bila kita mengurusi rumag atau bangunan yang menjadi penerima energi.\\\\
\begin{figure}
    \centering
    \includegraphics[width=0.5\linewidth]{Diagram simulasi.jpg}
    \caption{Prinsip dasar simulasi sistem bangunan}
    \label{fig:placeholder}
\end{figure}
\subsection{Karakteristik Iklim}
Karakteristik tipikal empat jenis iklim utama, ditinjau dari temperatur rata-rata, iradiansi surya harian, dan curah hujan rata-rata. Data iklim tahunan dinyatakan sebagai \textit{Test reference year} (TRY), yaitu data yang diambil dari tahun-tahun berbeda.
\subsection{Angin}
Data angin umumnya dinyataan dalam bentuk diagram 'mawar angin' yang menunjukkan frekuensi kemunculan angin pada 8 arah mata angin selama 1 bulan atau 1 tahun. Data ini digunakan utnuk membantu dalam membuat keputusan desain pasif untuk ventilasi udara mengikuti arah mata angin.
\subsection{Radiasi Surya}
Kuantitas radiasi surya dinyatakan dengan iradiansi ($W/m^2$), yaitu daya radiasi yang diterima suatu permukaan per satuan luas permukaan tersebut.
\subsection{Pulau Panas Perkotaan}
Wilayah perkotaan seringkali memilik temperatur rata-rata yang lebih tinggi daripada wilayah sekitarnya. Penyebab utamanya adalah properti termal dair permukaan tanah akibat adanya bangunan serta objek objek lain.

\end{document}