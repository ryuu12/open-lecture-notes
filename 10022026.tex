\documentclass{article}
\usepackage{graphicx} % Required for inserting images
\usepackage{tikz}
\usepackage{circuitikz}
\usepackage{pgfplots}

\title{Lecture Note Week 1}
\author{TF2201 Matematika Fisika 2}

\begin{document}

\maketitle
\section{Bilangan Kompleks}
Bilangan kompleks adalah bilangan yang memiliki bilangan imajiner.
\subsection{Bilangan Imajiner}
Bilangan imajiner didapatkan dari,
\begin{equation}
    x^2=-1
\end{equation}
\begin{equation}
    x=\sqrt{-1}
\end{equation}
Bilangan ini diberikan simbol $i$. Maka,
\begin{equation}
    i=x=\sqrt{-1}
\end{equation}
Di dalam bilangan kompleks, bilangan imajiner ditempatkan seperti berikut.
\begin{equation}
    z=(x, y)
\end{equation}
\begin{equation}
    z=x+iy
\end{equation}
Dimana $x$ merupakan bagian bilangan riil dari bilangan kompleks, dan $y$ merupakan bagian bilangan imajiner dari bilangan kompleks.
\subsection{Penjumlahan dan Perkalian}
Penjumlahan bilangan kompleks dapat dapat dilakukan seperti penjumlahan aritmatika biasa. Bila terdapat bilangan kompleks sebagai berikut.
\begin{equation}
    z_1=x_1+iy_1
\end{equation}
\begin{equation}
    z_2=x_2+iy_2
\end{equation}
Maka untuk melakukan penjumlahan adalah seperti berikut.
\begin{equation}
    z_1+z_2=(x_1+x_2, y_1+y_2)=(x_1+x_2)+i(y_1+y_2)
\end{equation}
Untuk melakukan perkalian, dapat dilakukan secara aritmatika biasa. Namun, untuk mempermudah, bentuk perkalian akan menjadi seperti ini.
\begin{equation}
    z_1\cdot z_2=(x_1x_2-y_1y_2,x_1y_2+x_2y_1)=(x_1x_2-y_1y_2)+i(x_1y_2+x_2y_1)
\end{equation}
\subsubsection{Contoh}
Diketahui nilai $z_1$ dan $z_2$ sebagai berikut.
\begin{itemize}
    \item $z_1=(8,3)$
    \item $z_2=(9,-2)$
\end{itemize}
Maka tentukan,
\begin{itemize}
    \item $z_1+z_2$
    \item $z_1\cdot z_2$
\end{itemize}
Untuk penjumlahan, maka gunakan persamaan (8). Nantinya, akan didapatkan hasil seperti ini.
\begin{equation}
    z_1+z_2=(8+9,3-2)=(8+9)+i(3-2)=17+i
\end{equation}
Untuk perkalian, gunakan persamaan (9). Nantinya, akan didapatkan hasil seperti ini.
\begin{equation}
    z_1\cdot z_2=(8\cdot 9-3\cdot(-2),8\cdot (-2)+9\cdot 3)=(8\cdot 9-3\cdot(-2))+i(8\cdot (-2)+9\cdot 3)
\end{equation}
\begin{equation}
    z_1\cdot z_2=(72+6)+i(-16+27)=78+11i
\end{equation}
\subsection{Pembagian}
Untuk melakukan pembagian, dapat digunakan bentuk seperti ini.
\begin{equation}
    \frac{z_1}{z_2}=\frac{x_1x_2+y_1y_2}{(x_2)^2+(y_2)^2}+i\frac{x_2y_1-x_1y_2}{(x_2)^2+(y_2)^2}
\end{equation}
\subsubsection{Contoh}
Diberikan nilai $z_1$ dan $z_2$ yang sama dengan contoh 1.2.1. Maka tentukan $\frac{z_1}{z_2}$.\\\\
Untuk menjawabnya, gunakan persamaan (13). Maka akan didapatkan nilai sebagai berikut.
\begin{equation}
    \frac{z_1}{z_2}=\frac{8\cdot 9+3\cdot (-2)}{9^2+(-2)^2}+i\left(\frac{9\cdot 3-8\cdot(-2)}{9^2+(-2)^2}\right)
\end{equation}
\begin{equation}
    \frac{z_1}{z_2}=\frac{72-6}{85}+i\frac{27+16}{85}=\frac{66}{85}+i\frac{43}{85}
\end{equation}
\subsection{Bentuk Polar}
Bilangan kompleks dapat juga ditulis dalam bentuk polar. Dengan memperhatikan $x$ dan $y$ sebagai berikut.
\begin{equation}
    x=r \cos \theta
\end{equation}
\begin{equation}
    y=r \sin \theta
\end{equation}
Maka, bentuk bilangan kompleks akan menjadi seperti berikut.
\begin{equation}
    z=(r \cos \theta)+i(r \sin \theta)
\end{equation}
\begin{equation}
    z=r(\cos \theta + \sin \theta)
\end{equation}
Nilai $r$ pada bentuk polar adalah nilai absolut dari $z$.
\begin{equation}
    r=|z|=\sqrt{x^2+y^2}
\end{equation}
Dimana absolut $z$ adalah perkalian antara $z$ dengan konjugatnya.
\begin{equation}
    r=\sqrt{z\bar{z}}
\end{equation}
Dalam bentuk polar, $\theta$ disebut sebagai argumen.
\begin{equation}
    \tan \theta=\frac{y}{z}
\end{equation}
\section{Bidang Kompleks}
Bidang kompleks berbentuk seperti bidang kartesius yang memiliki sumbu $x$ yang $y$. Namun, pada bidang kompleks, sumbu $x$ merupakan sumbu riil atau $R$, sementara sumbu $y$ merupakan sumbu imajiner atau $i$.\\\\
Dari bidang kompleks, kita dapat melakukan visualisasi dari sebuah bilang kompleks. Contohnya, apabila diketahui bilangan kompleks sebagai berikut,
\begin{equation}
    z=3+i7
\end{equation}
Maka, pembacaan pada bidang kompleks adalah angka 3 pada sumbu $x$ atau $R$, dan angka 7 pada sumbu $y$ atau sumbu $i$. Dikarenakan bilangan kompleks dapat ditempatkan pad bidang dua dimensi, maka bilangan kompleks dapat diperlakukan seperti vektor. Kita ambil contoh pada bentuk polar bilangan kompleks. Dari bentuk polar, diketahui bahwa,
\begin{itemize}
    \item $r$ adalah panjang vektor.
    \item $\theta$ adalah suduh yang dibentuk oleh vektor bilangan kompleks terhadap sumbu $x$.
    \item Nilai riil atau bagian $x$ dari $x+iy$ adalah nilai pada sumbu $x$.
    \item Nilai imajiner atau bagian $y$ dari $x+iy$ adalah nilai pada sumbu $y$.
\end{itemize}
\end{document}